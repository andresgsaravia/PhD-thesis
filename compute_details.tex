\chapter{Computational details}
\label{chap:comp_details}

In this appendix we give some details regarding the algorithms used for the calculations in this thesis as well as a couple of simple observations showing the adequacy of the size of the basis set used.
The algorithm used is released under a GNU General Public License version 3 (GPLv3) and, at the time of this writing, it is freely accessible on the Internet \cite{polaronsGitHub2014}.

\section{Algorithm}
\label{sec:algorithm}

The first iterations of our calculations were made with a QR algorithm  which is a full-matrix diagonalization procedure.
The particular implemenation was that provided by the \textit{Eigen} C++ package \cite{eigenweb}.
The choice of this algorithm was mainly due to its accessibility and ease of use. 
However, the hamiltonian matrix is sparse, that is, most of its elements are zero.
In fact, only about $\sim 0.03\%$ of them are different from zero.
Thus, it is more appropriate to use a diagonalization procedure optimized for sparse matrices.

Fortunately, there was sparse matrix support \cite{Bateman2006} in the popular Octave package \cite{Octave2014} introduced during the timeframe of this project so we could switch to that implementation in the later stages.
This implementation uses an interface to the ARPACK package \cite{LehoucqARPACK1997} and is able to find the lowest eigenvalues and eigenvectors of a sparse matrix with drastically inferior compuational resources compared to the QR algorithm.

The algorithm used for this thesis \cite{polaronsGitHub2014} is very simple.
First it builds a hamiltonian matrix from (\ref{eq:full-hamiltonian}) and afterwards it calls the sparse diagonalization routine implemented within Octave.
The only non-trivial aspect is the mapping from the quantum number $\ket{e_1,e_2,ir,R}$ (see section \ref{sec:hamiltonian-and-basis}) to an integer $l$ labelling the row/column in the matrix.
This mapping can be achieved as follows:
%
\begin{equation}
  \label{eq:label}
  l = e_1 + 3(e_2 - 1) + 9\ ir + 9\ R (N_{ir} +1)
\end{equation}
%
with $N_{ir}$ being the total number of infrared phonons under consideration.

\section{Convergence}
\label{sec:convergence}

The basis includes up to 30 harmonic Raman and 30 harmonic infrared phonons. 
The  truncation of the basis to a finite number of phonons could introduce innacuracies, however we found this choice to be an adecuate choice to ensure convergence in the few lowest energy states we are considering and electron-lattice couplings up to $\sim0.25$ eV\footnote{There is another numerical instability present in the calculation of the isotopic shift of the first excitation (see the top panel of figure \ref{fig:irIsot}) but this seems to be due to floating point errors in the calculations since all numbers are very close to zero.}.

\section{Numerical instabilities}
\label{sec:numericalInstabilities}

We performed an exact diagonalization of the Hamiltonian matrix with a basis of 8649 states using a QR algorithm \cite{eigenweb}\footnote{The hamiltonian (\ref{eq:full-hamiltonian}) is a sparse matrix, so a more efficient approach is being taken in subsequent calculations, in extensions of this model, using Lanczos algorithm. In this work we used a QR algorithm because it was readily available and provided a large enough basis set to ensure convergence.}.

% Maybe a plot here showing convergence in the energy of the ground state for different number of phonons?
