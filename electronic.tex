\chapter{Electronic excitations}
\label{chap:electronic}

We now turn our attention to the other kind of excitations present in the model hamiltonian (\ref{eq:full-hamiltonian}), namely, the \textit{electronic} excitations.
These excitations are those that, in the absence of charge-lattice coupling, arise from the $H_{el}$ term (\ref{eq:electronic-part}).
Since they do not have an opposing parity to the ground state, they are inactive in infrared and Raman spectroscopies but could be accessible by electronic probes such as ARPES\footnote{Angle Resolved Photo-Emission Spectroscopy}.
It should be noted however that ARPES probes the \textit{occupied} electronic states in a system and such an interpretation is difficult in our model.

Unlike $H_{ph}$, the electronic term $H_{el}$ is not infinite dimentional since it depends only on the position of the charges in the atomic sites. 
Using the labelling convention stated in (\ref{eq:basis-set}), $H_{el}$ can be expressed as the following 9x9 matrix:
%
\begin{equation}\label{eq:Hel-matrix} 
  H_{el} \doteq
  \left( 
    \begin{array}{ccccccccc} 
      U+2E_0 &\;\;t\;\;&\;\;0\;\;&\;\;t\;\;&0&\;\;0\;\;&\;\;0\;\;&\;\;0\;\;&0 \\
      t&0&t&0&t&0&0&0&0 \\
      0&t&2E_0 &0&0&t&0&0&0 \\
      t&0&0&0&t&0&t&0&0 \\
      0&t&0&t&U-2E_0 &t&0&t&0 \\
      0&0&t&0&t&0&0&0&t \\
      0&0&0&t&0&0&2E_0 &t&0 \\
      0&0&0&0&t&0&t&0&t \\
      0&0&0&0&0&t&0&t&U+2E_0  
    \end{array} 
  \right)
\end{equation}
%
This matrix is easily diagonalized to see that its first excited state has an energy of $\sim 1376$ cm$^{-1}$ above the ground state which is considerable higher than the phononic energies of 500 and 612.4 cm$^{-1}$.

Since $H_{el-ph}$ mixes electronic and phononic degrees of freedom, a change in the coupling parameter $\lambda_{ir}$ also changes the properties of these excitations.
Similar to the phononic excitations analyzed in the previous chapter these excitations do not remain eigenstates of $H_{el}$ for $\lambda_{ir}>0$ however, as discussed in section \ref{sec:classification}, in the middle coupling regime they remain mainly electronic in nature so we continue to refer them as such.

In Figure \ref{fig:electrSpectra} we show the energy of the three lowest electronic excitations as a function of $\lambda_{ir}$.
All energies show a monotonically decreasing behaviour with $\lambda_{ir}$, reminiscent of the phononic excitations in Figure \ref{fig:irSpectra} with little change in the small and strong coupling regimes but considerable change is in the middle coupling regime.
The first excitation starts at $\sim 1376$ cm$^{-1}$ for $\lambda_{ir}=0$ and converges to $\sim 1122$ cm$^{-1}$ for large $\lambda_{ir}$.
The electronic excitation with one additional infrared phonon also converges to this energy.
The presence of a Raman phonon rigidly increases the energies by $\omega_R=500$ cm$^{-1}$ for every $\lambda_{ir}$.
%
\begin{figure}[ht]
  \centering
  \input{images/electrSpectra}
  \caption[Energy of the electronic excitations as function of $\lambda_{ir}$.]
  {Energy of the electronic excitations as function of $\lambda_{ir}$. 
    The red line corresponds to an electronic excitation with zero phonons, the blue line with one infrared phonon and the green line with one Raman phonon.
    The vertical line is placed at the relevant value $\lambda_{ir}=0.1263$ eV.}
  \label{fig:electrSpectra}
\end{figure}

At $\lambda_{ir}=0.1263$ eV the first electronic excitation has the lower energy of $\sim 1282$ cm$^{-1}$ however the excitation with one additional infrared phonon has an energy of $\sim 1373$ cm$^{-1}$ which is very close to the $\sim 1376$ cm$^{-1}$ value for the first excitation at $\lambda_{ir}=0$.
Wa
\section{Partial charge localization}
\label{sec:partialChargeLocalization}

We use (\ref{eq:electronOccupation}) to find the probability $P(e_1,e_2)$ of finding an excitation with the first and second charges at sites $e_1$ and $e_2$ respectively where $e_i=1,3$ denotes a charge at the oxygen sites and $e_i=2$ at the copper site.
In Figure \ref{fig:electronicOccupations} we show the probabilites for one charge in an oxygen and the other in the copper (top panel) as well as both charges in the opposite oxygen sites (bottom panel).
The probabilities of finding both charges in the same site are very low, due to the large on-site Coulumb repulsion $U=7$ eV, so we omit a figure about them.
It should be noted however that the double occupancy probabilities remain in the $\sim 0.1$\% range for the ground and first infrared excitations but are zero in the electronic excitation for all $\lambda_{ir}$.
Since the cluster is symmetric some probabilities are equivalent: $P(1,2)=P(2,1)=P(2,3)=P(3,2)$, $P(1,3)=P(3,1)$ and $P(1,1)=P(3,3)$.

\begin{figure}[ht]
  \centering
  % GNUPLOT: LaTeX picture with Postscript
\begingroup
  \makeatletter
  \providecommand\color[2][]{%
    \GenericError{(gnuplot) \space\space\space\@spaces}{%
      Package color not loaded in conjunction with
      terminal option `colourtext'%
    }{See the gnuplot documentation for explanation.%
    }{Either use 'blacktext' in gnuplot or load the package
      color.sty in LaTeX.}%
    \renewcommand\color[2][]{}%
  }%
  \providecommand\includegraphics[2][]{%
    \GenericError{(gnuplot) \space\space\space\@spaces}{%
      Package graphicx or graphics not loaded%
    }{See the gnuplot documentation for explanation.%
    }{The gnuplot epslatex terminal needs graphicx.sty or graphics.sty.}%
    \renewcommand\includegraphics[2][]{}%
  }%
  \providecommand\rotatebox[2]{#2}%
  \@ifundefined{ifGPcolor}{%
    \newif\ifGPcolor
    \GPcolortrue
  }{}%
  \@ifundefined{ifGPblacktext}{%
    \newif\ifGPblacktext
    \GPblacktextfalse
  }{}%
  % define a \g@addto@macro without @ in the name:
  \let\gplgaddtomacro\g@addto@macro
  % define empty templates for all commands taking text:
  \gdef\gplbacktext{}%
  \gdef\gplfronttext{}%
  \makeatother
  \ifGPblacktext
    % no textcolor at all
    \def\colorrgb#1{}%
    \def\colorgray#1{}%
  \else
    % gray or color?
    \ifGPcolor
      \def\colorrgb#1{\color[rgb]{#1}}%
      \def\colorgray#1{\color[gray]{#1}}%
      \expandafter\def\csname LTw\endcsname{\color{white}}%
      \expandafter\def\csname LTb\endcsname{\color{black}}%
      \expandafter\def\csname LTa\endcsname{\color{black}}%
      \expandafter\def\csname LT0\endcsname{\color[rgb]{1,0,0}}%
      \expandafter\def\csname LT1\endcsname{\color[rgb]{0,1,0}}%
      \expandafter\def\csname LT2\endcsname{\color[rgb]{0,0,1}}%
      \expandafter\def\csname LT3\endcsname{\color[rgb]{1,0,1}}%
      \expandafter\def\csname LT4\endcsname{\color[rgb]{0,1,1}}%
      \expandafter\def\csname LT5\endcsname{\color[rgb]{1,1,0}}%
      \expandafter\def\csname LT6\endcsname{\color[rgb]{0,0,0}}%
      \expandafter\def\csname LT7\endcsname{\color[rgb]{1,0.3,0}}%
      \expandafter\def\csname LT8\endcsname{\color[rgb]{0.5,0.5,0.5}}%
    \else
      % gray
      \def\colorrgb#1{\color{black}}%
      \def\colorgray#1{\color[gray]{#1}}%
      \expandafter\def\csname LTw\endcsname{\color{white}}%
      \expandafter\def\csname LTb\endcsname{\color{black}}%
      \expandafter\def\csname LTa\endcsname{\color{black}}%
      \expandafter\def\csname LT0\endcsname{\color{black}}%
      \expandafter\def\csname LT1\endcsname{\color{black}}%
      \expandafter\def\csname LT2\endcsname{\color{black}}%
      \expandafter\def\csname LT3\endcsname{\color{black}}%
      \expandafter\def\csname LT4\endcsname{\color{black}}%
      \expandafter\def\csname LT5\endcsname{\color{black}}%
      \expandafter\def\csname LT6\endcsname{\color{black}}%
      \expandafter\def\csname LT7\endcsname{\color{black}}%
      \expandafter\def\csname LT8\endcsname{\color{black}}%
    \fi
  \fi
  \setlength{\unitlength}{0.0500bp}%
  \begin{picture}(5102.00,5668.00)%
    \gplgaddtomacro\gplbacktext{%
      \colorrgb{0.31,0.31,0.31}%
      \put(217,3384){\makebox(0,0)[r]{\strut{}\scriptsize 19}}%
      \colorrgb{0.31,0.31,0.31}%
      \put(217,3765){\makebox(0,0)[r]{\strut{}\scriptsize 20}}%
      \colorrgb{0.31,0.31,0.31}%
      \put(217,4145){\makebox(0,0)[r]{\strut{}\scriptsize 21}}%
      \colorrgb{0.31,0.31,0.31}%
      \put(217,4526){\makebox(0,0)[r]{\strut{}\scriptsize 22}}%
      \colorrgb{0.31,0.31,0.31}%
      \put(217,4906){\makebox(0,0)[r]{\strut{}\scriptsize 23}}%
      \colorrgb{0.31,0.31,0.31}%
      \put(217,5287){\makebox(0,0)[r]{\strut{}\scriptsize 24}}%
      \colorrgb{0.31,0.31,0.31}%
      \put(217,5667){\makebox(0,0)[r]{\strut{}\scriptsize 25}}%
      \colorrgb{0.31,0.31,0.31}%
      \put(396,3117){\makebox(0,0){\strut{}}}%
      \colorrgb{0.31,0.31,0.31}%
      \put(762,3117){\makebox(0,0){\strut{}}}%
      \colorrgb{0.31,0.31,0.31}%
      \put(1128,3117){\makebox(0,0){\strut{}}}%
      \colorrgb{0.31,0.31,0.31}%
      \put(1494,3117){\makebox(0,0){\strut{}}}%
      \colorrgb{0.31,0.31,0.31}%
      \put(1859,3117){\makebox(0,0){\strut{}}}%
      \colorrgb{0.31,0.31,0.31}%
      \put(2225,3117){\makebox(0,0){\strut{}}}%
      \colorrgb{0.31,0.31,0.31}%
      \put(2591,3117){\makebox(0,0){\strut{}}}%
      \colorrgb{0.31,0.31,0.31}%
      \put(2957,3117){\makebox(0,0){\strut{}}}%
      \colorrgb{0.31,0.31,0.31}%
      \put(3323,3117){\makebox(0,0){\strut{}}}%
      \colorrgb{0.31,0.31,0.31}%
      \put(3689,3117){\makebox(0,0){\strut{}}}%
      \colorrgb{0.31,0.31,0.31}%
      \put(4054,3117){\makebox(0,0){\strut{}}}%
      \colorrgb{0.31,0.31,0.31}%
      \put(4420,3117){\makebox(0,0){\strut{}}}%
      \colorrgb{0.31,0.31,0.31}%
      \put(4786,3117){\makebox(0,0){\strut{}}}%
      \csname LTb\endcsname%
      \put(-157,4525){\rotatebox{-270}{\makebox(0,0){\strut{}$P(1,2)$ (\%)}}}%
      \put(2682,3051){\makebox(0,0){\strut{}}}%
      \put(2774,5484){\makebox(0,0)[l]{\strut{}\scriptsize$\lambda_{ir}=0.1263$}}%
    }%
    \gplgaddtomacro\gplfronttext{%
    }%
    \gplgaddtomacro\gplbacktext{%
      \colorrgb{0.31,0.31,0.31}%
      \put(217,751){\makebox(0,0)[r]{\strut{}\scriptsize 0}}%
      \colorrgb{0.31,0.31,0.31}%
      \put(217,1145){\makebox(0,0)[r]{\strut{}\scriptsize 2}}%
      \colorrgb{0.31,0.31,0.31}%
      \put(217,1540){\makebox(0,0)[r]{\strut{}\scriptsize 4}}%
      \colorrgb{0.31,0.31,0.31}%
      \put(217,1934){\makebox(0,0)[r]{\strut{}\scriptsize 6}}%
      \colorrgb{0.31,0.31,0.31}%
      \put(217,2328){\makebox(0,0)[r]{\strut{}\scriptsize 8}}%
      \colorrgb{0.31,0.31,0.31}%
      \put(217,2723){\makebox(0,0)[r]{\strut{}\scriptsize 10}}%
      \colorrgb{0.31,0.31,0.31}%
      \put(217,3117){\makebox(0,0)[r]{\strut{}\scriptsize 12}}%
      \colorrgb{0.31,0.31,0.31}%
      \put(396,484){\makebox(0,0){\strut{}\scriptsize 0}}%
      \colorrgb{0.31,0.31,0.31}%
      \put(762,484){\makebox(0,0){\strut{}\scriptsize 0.02}}%
      \colorrgb{0.31,0.31,0.31}%
      \put(1128,484){\makebox(0,0){\strut{}\scriptsize 0.04}}%
      \colorrgb{0.31,0.31,0.31}%
      \put(1494,484){\makebox(0,0){\strut{}\scriptsize 0.06}}%
      \colorrgb{0.31,0.31,0.31}%
      \put(1859,484){\makebox(0,0){\strut{}\scriptsize 0.08}}%
      \colorrgb{0.31,0.31,0.31}%
      \put(2225,484){\makebox(0,0){\strut{}\scriptsize 0.1}}%
      \colorrgb{0.31,0.31,0.31}%
      \put(2591,484){\makebox(0,0){\strut{}\scriptsize 0.12}}%
      \colorrgb{0.31,0.31,0.31}%
      \put(2957,484){\makebox(0,0){\strut{}\scriptsize 0.14}}%
      \colorrgb{0.31,0.31,0.31}%
      \put(3323,484){\makebox(0,0){\strut{}\scriptsize 0.16}}%
      \colorrgb{0.31,0.31,0.31}%
      \put(3689,484){\makebox(0,0){\strut{}\scriptsize 0.18}}%
      \colorrgb{0.31,0.31,0.31}%
      \put(4054,484){\makebox(0,0){\strut{}\scriptsize 0.2}}%
      \colorrgb{0.31,0.31,0.31}%
      \put(4420,484){\makebox(0,0){\strut{}\scriptsize 0.22}}%
      \colorrgb{0.31,0.31,0.31}%
      \put(4786,484){\makebox(0,0){\strut{}\scriptsize 0.24}}%
      \csname LTb\endcsname%
      \put(-157,1934){\rotatebox{-270}{\makebox(0,0){\strut{}$P(1,3)$ (\%)}}}%
      \put(2682,154){\makebox(0,0){\strut{}$\lambda_{ir}$ (eV)}}%
      \put(2774,2928){\makebox(0,0)[l]{\strut{}\scriptsize$\lambda_{ir}=0.1263$}}%
    }%
    \gplgaddtomacro\gplfronttext{%
    }%
    \gplbacktext
    \put(0,0){\includegraphics{images/electronicOccupations}}%
    \gplfronttext
  \end{picture}%
\endgroup

  \caption[Probability of finding the system with one charge in an oxygen and the other at the copper and both charges at opposing oxygen sites.]
  {Probability of finding the system with one charge in an oxygen and the other at the copper (top panel) and both charges at opposing oxygen sites (bottom panel) as a fucntion of $\lambda_{ir}$.
  The vertical line marks the relevant $\lambda_{ir}=0.1263$ eV value.}
  \label{fig:electronicOccupations}
\end{figure}

For all three eigenstates the general trend is similar; there is a monotonic increase in the probability of finding one charge in the oxygen and the other in the copper sites while the probability of both charges being in the opposite oxygen sites decreases.
This can be interpreted as a partial charge localization in the cluster, although there is still some charge movement \cite{GarciaSaraviaOrtizdeMontellano2013}.

\section{Projection into phonon coordinates}
\label{sec:elPhononProj}

We also investigated the projection into phonon coordinates for the first electronic excitation in the model.
In Figure \ref{fig:phononProjElectr} we show these projections for $\lambda_{ir}=$0, 0.1263 and 0.25 eV corresponding to values in the weak, middle and strong coupling regimes.
Similar to the other excitations, the projection along $u_R$ remains unchanged in the form of a gaussian function.
At $\lambda_{ir}=0$ eV the projection along $u_{ir}$ has also a gaussian shape however, for the coupling value that reproduces the cluster distortion ($\lambda_{ir}=0.1263$ eV), it develops a double peak structure reminiscent of the ground state in Figure \ref{fig:phononProjGrdPol} although with more defined peaks.
For large $\lambda_{ir}$ the two peaks are completely separated each one similar to the $\lambda_{ir}=0$ eV case but displaced in origin.
This behaviour at large coupling values is similar to what was observed for all the other excitations in which the uncoupled scenario is recovered with a displaced origin.

\begin{figure}[ht]
  \centering
  \input{images/phononProjElectr}
  \caption[Projection into phonon coordinates of the electronic excitation.]
  {Projection into phonon coordinates of the electronic excitation for three representative $\lambda_{ir}$ values.}
  \label{fig:phononProjElectr}
\end{figure}

\section{Isotopic shift}
\label{sec:elIsotShift}

We also investigated the effect of an oxygen isotopic change $^{16}$O$\rightarrow ^{18}$O in the eigenvalues of the electronic excitations.
In Figure \ref{fig:electrIsot} we show a plot of the isotopic shift $\Delta_g$ as defined in (\ref{eq:isot-shift-def-grd}) for the first electronic excitation as well as the excitation with an additional infrared phonon for a range of $\lambda_{ir}$ values.
Since the electronic excitations are unaffected by changes in the phonon energies when $\lambda_{ir}=0$ eV, the isotopic shift of the first electronic excitation is zero and the shift for the excitation with one additional infared phonon is what could be expected from the harmonic behaviour of the system.
Interestingly, the isotopic shift decreases to a minimum close to the relevant $\lambda_{ir}=0.1263$ eV value and then increases again becoming almost linear for $\lambda_{ir}>0.18$ eV.
%
\begin{figure}[ht]
  \centering
  % GNUPLOT: LaTeX picture with Postscript
\begingroup
  \makeatletter
  \providecommand\color[2][]{%
    \GenericError{(gnuplot) \space\space\space\@spaces}{%
      Package color not loaded in conjunction with
      terminal option `colourtext'%
    }{See the gnuplot documentation for explanation.%
    }{Either use 'blacktext' in gnuplot or load the package
      color.sty in LaTeX.}%
    \renewcommand\color[2][]{}%
  }%
  \providecommand\includegraphics[2][]{%
    \GenericError{(gnuplot) \space\space\space\@spaces}{%
      Package graphicx or graphics not loaded%
    }{See the gnuplot documentation for explanation.%
    }{The gnuplot epslatex terminal needs graphicx.sty or graphics.sty.}%
    \renewcommand\includegraphics[2][]{}%
  }%
  \providecommand\rotatebox[2]{#2}%
  \@ifundefined{ifGPcolor}{%
    \newif\ifGPcolor
    \GPcolortrue
  }{}%
  \@ifundefined{ifGPblacktext}{%
    \newif\ifGPblacktext
    \GPblacktextfalse
  }{}%
  % define a \g@addto@macro without @ in the name:
  \let\gplgaddtomacro\g@addto@macro
  % define empty templates for all commands taking text:
  \gdef\gplbacktext{}%
  \gdef\gplfronttext{}%
  \makeatother
  \ifGPblacktext
    % no textcolor at all
    \def\colorrgb#1{}%
    \def\colorgray#1{}%
  \else
    % gray or color?
    \ifGPcolor
      \def\colorrgb#1{\color[rgb]{#1}}%
      \def\colorgray#1{\color[gray]{#1}}%
      \expandafter\def\csname LTw\endcsname{\color{white}}%
      \expandafter\def\csname LTb\endcsname{\color{black}}%
      \expandafter\def\csname LTa\endcsname{\color{black}}%
      \expandafter\def\csname LT0\endcsname{\color[rgb]{1,0,0}}%
      \expandafter\def\csname LT1\endcsname{\color[rgb]{0,1,0}}%
      \expandafter\def\csname LT2\endcsname{\color[rgb]{0,0,1}}%
      \expandafter\def\csname LT3\endcsname{\color[rgb]{1,0,1}}%
      \expandafter\def\csname LT4\endcsname{\color[rgb]{0,1,1}}%
      \expandafter\def\csname LT5\endcsname{\color[rgb]{1,1,0}}%
      \expandafter\def\csname LT6\endcsname{\color[rgb]{0,0,0}}%
      \expandafter\def\csname LT7\endcsname{\color[rgb]{1,0.3,0}}%
      \expandafter\def\csname LT8\endcsname{\color[rgb]{0.5,0.5,0.5}}%
    \else
      % gray
      \def\colorrgb#1{\color{black}}%
      \def\colorgray#1{\color[gray]{#1}}%
      \expandafter\def\csname LTw\endcsname{\color{white}}%
      \expandafter\def\csname LTb\endcsname{\color{black}}%
      \expandafter\def\csname LTa\endcsname{\color{black}}%
      \expandafter\def\csname LT0\endcsname{\color{black}}%
      \expandafter\def\csname LT1\endcsname{\color{black}}%
      \expandafter\def\csname LT2\endcsname{\color{black}}%
      \expandafter\def\csname LT3\endcsname{\color{black}}%
      \expandafter\def\csname LT4\endcsname{\color{black}}%
      \expandafter\def\csname LT5\endcsname{\color{black}}%
      \expandafter\def\csname LT6\endcsname{\color{black}}%
      \expandafter\def\csname LT7\endcsname{\color{black}}%
      \expandafter\def\csname LT8\endcsname{\color{black}}%
    \fi
  \fi
  \setlength{\unitlength}{0.0500bp}%
  \begin{picture}(6802.00,4534.00)%
    \gplgaddtomacro\gplbacktext{%
      \colorrgb{0.31,0.31,0.31}%
      \put(990,751){\makebox(0,0)[r]{\strut{}\scriptsize -2}}%
      \colorrgb{0.31,0.31,0.31}%
      \put(990,1191){\makebox(0,0)[r]{\strut{}\scriptsize -1.5}}%
      \colorrgb{0.31,0.31,0.31}%
      \put(990,1631){\makebox(0,0)[r]{\strut{}\scriptsize -1}}%
      \colorrgb{0.31,0.31,0.31}%
      \put(990,2070){\makebox(0,0)[r]{\strut{}\scriptsize -0.5}}%
      \colorrgb{0.31,0.31,0.31}%
      \put(990,2510){\makebox(0,0)[r]{\strut{}\scriptsize 0}}%
      \colorrgb{0.31,0.31,0.31}%
      \put(990,2950){\makebox(0,0)[r]{\strut{}\scriptsize 0.5}}%
      \colorrgb{0.31,0.31,0.31}%
      \put(990,3390){\makebox(0,0)[r]{\strut{}\scriptsize 1}}%
      \colorrgb{0.31,0.31,0.31}%
      \put(990,3829){\makebox(0,0)[r]{\strut{}\scriptsize 1.5}}%
      \colorrgb{0.31,0.31,0.31}%
      \put(990,4269){\makebox(0,0)[r]{\strut{}\scriptsize 2}}%
      \colorrgb{0.31,0.31,0.31}%
      \put(1169,484){\makebox(0,0){\strut{}\scriptsize 0}}%
      \colorrgb{0.31,0.31,0.31}%
      \put(1588,484){\makebox(0,0){\strut{}\scriptsize 0.02}}%
      \colorrgb{0.31,0.31,0.31}%
      \put(2007,484){\makebox(0,0){\strut{}\scriptsize 0.04}}%
      \colorrgb{0.31,0.31,0.31}%
      \put(2426,484){\makebox(0,0){\strut{}\scriptsize 0.06}}%
      \colorrgb{0.31,0.31,0.31}%
      \put(2845,484){\makebox(0,0){\strut{}\scriptsize 0.08}}%
      \colorrgb{0.31,0.31,0.31}%
      \put(3263,484){\makebox(0,0){\strut{}\scriptsize 0.1}}%
      \colorrgb{0.31,0.31,0.31}%
      \put(3682,484){\makebox(0,0){\strut{}\scriptsize 0.12}}%
      \colorrgb{0.31,0.31,0.31}%
      \put(4101,484){\makebox(0,0){\strut{}\scriptsize 0.14}}%
      \colorrgb{0.31,0.31,0.31}%
      \put(4520,484){\makebox(0,0){\strut{}\scriptsize 0.16}}%
      \colorrgb{0.31,0.31,0.31}%
      \put(4939,484){\makebox(0,0){\strut{}\scriptsize 0.18}}%
      \colorrgb{0.31,0.31,0.31}%
      \put(5358,484){\makebox(0,0){\strut{}\scriptsize 0.2}}%
      \colorrgb{0.31,0.31,0.31}%
      \put(5777,484){\makebox(0,0){\strut{}\scriptsize 0.22}}%
      \colorrgb{0.31,0.31,0.31}%
      \put(6196,484){\makebox(0,0){\strut{}\scriptsize 0.24}}%
      \csname LTb\endcsname%
      \put(352,2510){\rotatebox{-270}{\makebox(0,0){\strut{}$\Delta_i$}}}%
      \put(3787,154){\makebox(0,0){\strut{}$\lambda_{ir}$ (eV)}}%
      \put(3892,3988){\makebox(0,0)[l]{\strut{}\scriptsize$\lambda_{ir}=0.1263$}}%
    }%
    \gplgaddtomacro\gplfronttext{%
    }%
    \gplbacktext
    \put(0,0){\includegraphics{images/electrIsot}}%
    \gplfronttext
  \end{picture}%
\endgroup

  \caption[Isotopic shift of the electronic excitations as a function of $\lambda_{ir}$.]
  {Isotopic shifts of the electronic excitations as a function of $\lambda_{ir}$. 
    The red line corresponds to an electronic excitation with zero phonons, the blue line with one infrared phonon and the green line with one Raman phonon.
    The vertical line is placed at the relevant value $\lambda_{ir}=0.1263$ eV.}
  \label{fig:electrIsot}
\end{figure}

For both excitatons shown in Figure \ref{fig:electrIsot} the isotopic shift becomes negative in the middle coupling regime similarly to the first infrared excitation at these values (see top panel of Figure \ref{fig:irIsot}) although with a reduced magnitude.
In contrast, for larger $\lambda_{ir}$ these shifts seem to increase indefinetly not stabilizing to any value.
Although the calculations seem to converge well in the $\lambda_{ir}$ range we are considering, we should note that there could be some convergence issues at large $\lambda_{ir}$ due to the truncation of the basis.

As mentioned section \ref{sec:isotopic_effects}, the $^{16}$O$\rightarrow ^{18}$O isotopic effect on electronic excitations has been experimentally explored by ARPES.
Gweon at al. \cite{Gweon2004} found a different isotope effect for nodal and antinodal excitations in optimally doped Bi$_2$Sr$_2$CaCu$_2$O$_{8+\delta}$. 
They found energy differences ranging approximately from -15 to 40 meV for nodal and antinodal exciations respectively.
However, a later study by Douglas et al. \cite{Douglas2007} found a much smaller, and possbily neglible, isotope effect ranging from $-0.9\pm 0.4$ to $2 \pm 3$ meV.

The interpretation of the electronic exciations observed by ARPES in terms of our simple model hamiltonian is not straightforward, however, we also find positive and negative isotopic effects that, in our case, corresponds to undistorted and distorted O(4)-Cu(1)-O(4) clusters respectively.
In contrast with the work of Gweon et al. we predict a considerable smaller isotopic effect around $\pm 1$\%, which is in the 1-2 meV range.
Although Douglas et al. claim to have observed a lack of isotopic effects on the electronic excitations, our prediction falls well into their experimental uncertainty prompting for more accurate ARPES studies in these samples.