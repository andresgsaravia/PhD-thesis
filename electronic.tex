\chapter{Electronic excitations}
\label{chap:electronic}

We now turn our attention to the other kind of excitations present in the model hamiltonian (\ref{eq:full-hamiltonian}), namely, the \textit{electronic} excitations that, in the absence of charge-lattice coupling, arise from the $H_{el}$ term (\ref{eq:electronic-part}).
These excitations are inactive in infrared and Raman spectroscopies but could be accessible by electronic probes.

Unlike $H_{ph}$, the electronic term $H_{el}$ is not infinite dimentional since it depends only on the position of the charges in the atomic sites. 
Using the labelling convention stated in (\ref{eq:basis-set}), it can be expressed as the following 9x9 matrix:
%
\begin{equation}\label{eq:Hel-matrix} \left( \begin{array}{ccccccccc} 
U+2E_0 &\;\;t\;\;&\;\;0\;\;&\;\;t\;\;&0&\;\;0\;\;&\;\;0\;\;&\;\;0\;\;&0 \\
t&0&t&0&t&0&0&0&0 \\
0&t&2E_0 &0&0&t&0&0&0 \\
t&0&0&0&t&0&t&0&0 \\
0&t&0&t&U-2E_0 &t&0&t&0 \\
0&0&t&0&t&0&0&0&t \\
0&0&0&t&0&0&2E_0 &t&0 \\
0&0&0&0&t&0&t&0&t \\
0&0&0&0&0&t&0&t&U+2E_0  \end{array} \right)\end{equation}
%
This matrix is easily diagonalized to see that its first excited state has an energy of $\sim 1376$ cm$^{-1}$ above the ground state which is a considerable higher than the phononic energies of 500 and 612.4 cm$^{-1}$.

Since $H_{el-ph}$ mixes electronic and phononic degrees of freedom, a change in the coupling parameter $\lambda_{ir}$ also changes the properties of these excitations.
Similar to the phononic excitations analyzed in the previous chapter, for $\lambda_{ir}>0$, these excitations do not remain eigenstates of $H_{el}$ however, as discussed in section \ref{sec:classification}, in the middle coupling regime they remain mainly electronic in nature so we continue to refer them as such.

In Figure \ref{fig:electrSpectra} we show the energy of the three lowest electronic excitations as a function of $\lambda_{ir}$.
All energies show a monotonically decreasing behaviour with $\lambda_{ir}$, reminiscent of the phononic excitations in Figure \ref{fig:irSpectra} with little change in the small and strong coupling regimes but considerable change is in the middle coupling regime.
The first excitation starts at $\sim 1376$ cm$^{-1}$ for $\lambda_{ir}=0$ and converges to $\sim 1122$ cm$^{-1}$ for large $\lambda_{ir}$.
The electronic excitation with one additional infrared phonon also converges to this energy and the presence of a Raman phonon only increases the energy by $\omega_R=500$ cm$^{-1}$ for every $\lambda_{ir}$.
%
\begin{figure}[ht]
  \centering
  \input{images/electrSpectra}
  \caption[Energy of the electronic excitations as function of $\lambda_{ir}$.]
  {Energy of the electronic excitations as function of $\lambda_{ir}$. 
    The red line corresponds to an electronic excitation with zero phonons, the blue line with one infrared phonon and the green line with one Raman phonon.
    The vertical line is placed at the relevant value $\lambda_{ir}=0.1263$ eV.}
  \label{fig:electrSpectra}
\end{figure}

At $\lambda_{ir}=0.1263$ eV the first electronic excitation has the lower energy of $\sim 1282$ cm$^{-1}$ however the excitation with one additional infrared phonon has an energy of $\sim 1373$ cm$^{-1}$ which is very close to the $\sim 1376$ cm$^{-1}$ value for the first excitation at $\lambda_{ir}=0$.

\section{Projection into definite electronic occupation states}

Since we are using basis states with definite electron occupancy, from the eigenvectors of the (\ref{eq:Hel-matrix}) matrix we can directly see the projections into definite electronic occupation states. The following table summarizes those values omitting equivalent basis states:

\noindent\begin{tabular}{| c | c | c | c | c | c | c | c | c | c |}
\hline
Energy (cm$^{-1}$) & 0.0 & 1376.42 & 3825.49 & 4325.12 & 14897.84 & 15329.82 & 53933.30 & 69272.75 & 69309.3 \\
\hline
$\uparrow \downarrow \ - \ -$ & 0.00276 & 0.00000 & 0.00382 & 0.00000 & 0.00090 & 0.00000 & 0.00179 & 0.49618 & 0.49455 \\
$\uparrow\  \downarrow \ -$ & 0.20130 & 0.19717 & 0.24809 & 0.25000 & 0.04045 & 0.05283 & 0.00606 & 0.00191 & 0.00219 \\
$\uparrow \ - \ \downarrow$ & 0.08522 & 0.10566 & 0.00000 & 0.00000 & 0.41451 & 0.39434 & 0.00023 & 0.00000 & 0.00000 \\
$ - \ \uparrow \downarrow \ -$ & 0.01884 & 0.00000 & 0.00000 & 0.00000 & 0.00736 & 0.00000 & 0.97174 & 0.00000 & 0.00205 \\
\hline
\end{tabular}

We noticed\cite{GarciaSaraviaOrtizdeMontellano2013} that, for the first \textit{electronic} excitation, the projection into states with an electron in each oxygen decreases with an increasing $\lambda_{ir}$ suggesting partial charge localization.

\section{Projection into phonon coordinates}

\begin{figure}[ht]
  \centering
  \input{images/phononProjElectr}
  \caption{Electronic state projected into phonon coordinates for three representative electron-lattice coupling ($\lambda_{ir}$) values.}
  \label{fig:phononProjElectr}
\end{figure}

\section{Isotopic shift}

\begin{figure}[ht]
  \centering
  % GNUPLOT: LaTeX picture with Postscript
\begingroup
  \makeatletter
  \providecommand\color[2][]{%
    \GenericError{(gnuplot) \space\space\space\@spaces}{%
      Package color not loaded in conjunction with
      terminal option `colourtext'%
    }{See the gnuplot documentation for explanation.%
    }{Either use 'blacktext' in gnuplot or load the package
      color.sty in LaTeX.}%
    \renewcommand\color[2][]{}%
  }%
  \providecommand\includegraphics[2][]{%
    \GenericError{(gnuplot) \space\space\space\@spaces}{%
      Package graphicx or graphics not loaded%
    }{See the gnuplot documentation for explanation.%
    }{The gnuplot epslatex terminal needs graphicx.sty or graphics.sty.}%
    \renewcommand\includegraphics[2][]{}%
  }%
  \providecommand\rotatebox[2]{#2}%
  \@ifundefined{ifGPcolor}{%
    \newif\ifGPcolor
    \GPcolortrue
  }{}%
  \@ifundefined{ifGPblacktext}{%
    \newif\ifGPblacktext
    \GPblacktextfalse
  }{}%
  % define a \g@addto@macro without @ in the name:
  \let\gplgaddtomacro\g@addto@macro
  % define empty templates for all commands taking text:
  \gdef\gplbacktext{}%
  \gdef\gplfronttext{}%
  \makeatother
  \ifGPblacktext
    % no textcolor at all
    \def\colorrgb#1{}%
    \def\colorgray#1{}%
  \else
    % gray or color?
    \ifGPcolor
      \def\colorrgb#1{\color[rgb]{#1}}%
      \def\colorgray#1{\color[gray]{#1}}%
      \expandafter\def\csname LTw\endcsname{\color{white}}%
      \expandafter\def\csname LTb\endcsname{\color{black}}%
      \expandafter\def\csname LTa\endcsname{\color{black}}%
      \expandafter\def\csname LT0\endcsname{\color[rgb]{1,0,0}}%
      \expandafter\def\csname LT1\endcsname{\color[rgb]{0,1,0}}%
      \expandafter\def\csname LT2\endcsname{\color[rgb]{0,0,1}}%
      \expandafter\def\csname LT3\endcsname{\color[rgb]{1,0,1}}%
      \expandafter\def\csname LT4\endcsname{\color[rgb]{0,1,1}}%
      \expandafter\def\csname LT5\endcsname{\color[rgb]{1,1,0}}%
      \expandafter\def\csname LT6\endcsname{\color[rgb]{0,0,0}}%
      \expandafter\def\csname LT7\endcsname{\color[rgb]{1,0.3,0}}%
      \expandafter\def\csname LT8\endcsname{\color[rgb]{0.5,0.5,0.5}}%
    \else
      % gray
      \def\colorrgb#1{\color{black}}%
      \def\colorgray#1{\color[gray]{#1}}%
      \expandafter\def\csname LTw\endcsname{\color{white}}%
      \expandafter\def\csname LTb\endcsname{\color{black}}%
      \expandafter\def\csname LTa\endcsname{\color{black}}%
      \expandafter\def\csname LT0\endcsname{\color{black}}%
      \expandafter\def\csname LT1\endcsname{\color{black}}%
      \expandafter\def\csname LT2\endcsname{\color{black}}%
      \expandafter\def\csname LT3\endcsname{\color{black}}%
      \expandafter\def\csname LT4\endcsname{\color{black}}%
      \expandafter\def\csname LT5\endcsname{\color{black}}%
      \expandafter\def\csname LT6\endcsname{\color{black}}%
      \expandafter\def\csname LT7\endcsname{\color{black}}%
      \expandafter\def\csname LT8\endcsname{\color{black}}%
    \fi
  \fi
  \setlength{\unitlength}{0.0500bp}%
  \begin{picture}(6802.00,4534.00)%
    \gplgaddtomacro\gplbacktext{%
      \colorrgb{0.31,0.31,0.31}%
      \put(990,751){\makebox(0,0)[r]{\strut{}\scriptsize -2}}%
      \colorrgb{0.31,0.31,0.31}%
      \put(990,1191){\makebox(0,0)[r]{\strut{}\scriptsize -1.5}}%
      \colorrgb{0.31,0.31,0.31}%
      \put(990,1631){\makebox(0,0)[r]{\strut{}\scriptsize -1}}%
      \colorrgb{0.31,0.31,0.31}%
      \put(990,2070){\makebox(0,0)[r]{\strut{}\scriptsize -0.5}}%
      \colorrgb{0.31,0.31,0.31}%
      \put(990,2510){\makebox(0,0)[r]{\strut{}\scriptsize 0}}%
      \colorrgb{0.31,0.31,0.31}%
      \put(990,2950){\makebox(0,0)[r]{\strut{}\scriptsize 0.5}}%
      \colorrgb{0.31,0.31,0.31}%
      \put(990,3390){\makebox(0,0)[r]{\strut{}\scriptsize 1}}%
      \colorrgb{0.31,0.31,0.31}%
      \put(990,3829){\makebox(0,0)[r]{\strut{}\scriptsize 1.5}}%
      \colorrgb{0.31,0.31,0.31}%
      \put(990,4269){\makebox(0,0)[r]{\strut{}\scriptsize 2}}%
      \colorrgb{0.31,0.31,0.31}%
      \put(1169,484){\makebox(0,0){\strut{}\scriptsize 0}}%
      \colorrgb{0.31,0.31,0.31}%
      \put(1588,484){\makebox(0,0){\strut{}\scriptsize 0.02}}%
      \colorrgb{0.31,0.31,0.31}%
      \put(2007,484){\makebox(0,0){\strut{}\scriptsize 0.04}}%
      \colorrgb{0.31,0.31,0.31}%
      \put(2426,484){\makebox(0,0){\strut{}\scriptsize 0.06}}%
      \colorrgb{0.31,0.31,0.31}%
      \put(2845,484){\makebox(0,0){\strut{}\scriptsize 0.08}}%
      \colorrgb{0.31,0.31,0.31}%
      \put(3263,484){\makebox(0,0){\strut{}\scriptsize 0.1}}%
      \colorrgb{0.31,0.31,0.31}%
      \put(3682,484){\makebox(0,0){\strut{}\scriptsize 0.12}}%
      \colorrgb{0.31,0.31,0.31}%
      \put(4101,484){\makebox(0,0){\strut{}\scriptsize 0.14}}%
      \colorrgb{0.31,0.31,0.31}%
      \put(4520,484){\makebox(0,0){\strut{}\scriptsize 0.16}}%
      \colorrgb{0.31,0.31,0.31}%
      \put(4939,484){\makebox(0,0){\strut{}\scriptsize 0.18}}%
      \colorrgb{0.31,0.31,0.31}%
      \put(5358,484){\makebox(0,0){\strut{}\scriptsize 0.2}}%
      \colorrgb{0.31,0.31,0.31}%
      \put(5777,484){\makebox(0,0){\strut{}\scriptsize 0.22}}%
      \colorrgb{0.31,0.31,0.31}%
      \put(6196,484){\makebox(0,0){\strut{}\scriptsize 0.24}}%
      \csname LTb\endcsname%
      \put(352,2510){\rotatebox{-270}{\makebox(0,0){\strut{}$\Delta_i$}}}%
      \put(3787,154){\makebox(0,0){\strut{}$\lambda_{ir}$ (eV)}}%
      \put(3892,3988){\makebox(0,0)[l]{\strut{}\scriptsize$\lambda_{ir}=0.1263$}}%
    }%
    \gplgaddtomacro\gplfronttext{%
    }%
    \gplbacktext
    \put(0,0){\includegraphics{images/electrIsot}}%
    \gplfronttext
  \end{picture}%
\endgroup

  \caption[Isotopic shift of the electronic excitations as a function of $\lambda_{ir}$.]
  {Isotopic shifts of the electronic excitations as a function of $\lambda_{ir}$. 
    The red line corresponds to an electronic excitation with zero phonons, the blue line with one infrared phonon and the green line with one Raman phonon.
    The vertical line is placed at the relevant value $\lambda_{ir}=0.1263$ eV.}
  \label{fig:electrIsot}
\end{figure}

