\chapter{Discussion and conclusions}
\label{chap:conclusions}

In this thesis we reviewed the evidence for the appearance of a dynamically inhomogeneous ground state in the pseudogap region. 
In this state there are bipolaronic objects which form at a characteristic temperature T*. 
The manifestations of these objects in the lattice appear as dynamical local lattice distortions in regions in which the charge and lattice motion become correlated. 
The excitations of these bipolaronic objects exhibit different isotopic shifts depending on the nature of the excitation ranging from large negative to quasi-harmonic shifts. 
These peculiar shifts are consistent with experimental observations and might explain discrepancies in values determined by different techniques, as they probe different excitations depending on the time and spatial resolution of the specific technique. 
However, the role in the pairing of free fermions of these bipolaronic excitations, and enhancement of T$_c$ is not yet known. 
In this spirit, some models have considered the interaction between fermionic \textit{pairs} and bipolaronic bosonic objects \cite{Bussmann-Holder2005,Mihailovic2001,Bar-Yam1991,doi:10.1142/S0217979200003812} explaining several properties of the normal state in this pseudogap region. 

\section{Validity of the Born-Oppenheimer approximation}

It is interesting to note that, although the isotopic shifts of the ground state and lowest excitations can be either positive or negative, they all exhibit the strongest variations in the intermediate coupling regime where the eigenstates in the system are mixed and it is not possible to separate them into an \textit{electronic} and a \textit{lattice} part.
It is in this intermediate regime, where the charge and lattice movements are intrinsically correlated, that the Born-Oppenheimer approximation ceases to be valid.
An exact treatment, such as the one used in this work, is free from such an approximation, however it is computationally expensive and  only relatively small systems can feasibly be explored at the moment.

The importance of the intermediate coupling regime suggests a characteristic dynamical scale for the polaronic features in the cuprate superconductors. 
Other transition metals oxides as manganites and nickelates exhibit similar local lattice distortions however, the size of the distortion is larger in manganites ($\sim 0.2$ \AA) \cite{Tyson1996} and smaller in nickelates ($\sim 0.05$ \AA) \cite{Acosta-Alejandro2008} resulting also in different time scales. 
It is enticing to relate the particular dynamical time scale of the local lattice distortions with the presence of high-temperature superconductivity in cuprates but not in other transition metal oxides thus highlighting the relevance of model hamiltonians not relying on the adiabatic and anti-adiabatic approximations.

\section{Multicomponent superconductivity}

Finally, we would like to discuss the possible relevance of the bipolaronic behavior, derived from the simple Peierls-Hubbard Hamiltonian model (\ref{eq:full-hamiltonian}) to high temperature superconductivity. 
The observation of a Fermi surface in photoemission experiments \cite{Ding1996,Hussey2003} implies that, in addition to the bipolaronic objects (of bosonic character), there are also quasi-free fermions. 
As a function of doping the ratio of these two kinds of objects varies, yielding different characterizations of the ground state starting as an antiferromagnetic Mott insulator at zero doping, an inhomogeneous pseudogap phase where at least two different type of carriers coexist and ending in a Fermi-liquid metal in the overdoped region of the phase diagram. 
The fact that the highest T$_c$  is realized in this pseudogap region suggests the importance of the bipolaron bosonic objects. 
However, their role in the pairing of free fermions and enhancement of T$_c$ is not yet known. 
In order to understand this role, a natural extension of the model treated here is to couple the Halmiltonian  (\ref{eq:full-hamiltonian}) to an independent a free fermion system $H_A$ and find the corresponding excitations allowing exchange of the fermions involved in the bipolaronic objects with those in the free fermion part. 
In the spirit of the exact treatment of the bipolaronic subsystem, the simplest interaction between mobile fermions and charges forming part of a bipolaron is a linear hopping term. 
Such a term would allow the transfer of a single fermion from the mobile fermion to the bipolaron subsystems. 
In principle, in an exact treatment like this, the possibility of pair hopping would be implicitly included. 
That is not the case in a perturbative treatment, in which the single particle exchange (hopping) can be folded into the pair exchange (see \cite{Bar-Yam1991}).
If we consider the simplest free fermion system consisting only on two non interacting fermions in real space, with creation and anihilation operators $a^\dagger$ and $a$ respectively, the coupling of (\ref{eq:full-hamiltonian}) to this sytem ($H_A$) can be written as

\begin{equation}
  \label{eq:multicomponent}
  H_A = 
  E_A \sum_{\sigma, k=1,2} m_{\sigma, k} + 
  t_A \sum_{\sigma} (a_{\sigma, 1}^\dagger a_{\sigma, 2} + a_{\sigma,2}^\dagger a_{\sigma, 1}) + 
  \lambda_A \sum_{i,k,\sigma} \left( c_{i,\sigma}^\dagger a_{k,\sigma} + a_{k,\sigma}^\dagger c_{i,\sigma} \right) 
\end{equation}
% 
with $i=1,2,3$ labeling the sites in the CuO$_2$ cluster and $k=1,2$ the two sites for the free fermions. 
Here $m_{\sigma, k} = a_{\sigma, k}^\dagger a_{\sigma, k}$ is the number operator in the free fermion subsystem, $E_A$ its site energy and $\lambda_A$ parametrizes its interaction with the CuO$_2$ cluster.
This approach still allows us to use an exact treatment, albeit missing the possible extended nature of the free fermion states. 
If a perturbative approach was used instead it could be possible to replace the free fermion subsystem by a full band in $k$-space. 
In addition, an onsite Coulomb repulsion term could be added to the free electron subsystem and still feasibly perform an exact diagonalization. 
In this case the basis set, considering the same number of phonons, is two orders of magnitude larger than the one used for the original Peirles-Hubbard Hamiltonian (\ref{eq:full-hamiltonian}) but still within reach of modern computational resources.
 
In this model an increased projection of a given excitation on a double occupancy basis state of the free fermion part would be a signature of pairing mediated by the bipolaronic objects in the CuO$_2$ cluster. 
Additionally, the role played by the antiferromagnetic background present at zero doping and its interaction with both the fermionic carriers and polaronic objects has to be addressed.
