% Aknowledgements

\cleardoublepage
\phantomsection
\addcontentsline{toc}{chapter}{Aknowledgments}

\begin{center}
\textbf{\large Agradecimientos}
\end{center}
\vspace{3cm}

\begin{itemize}
    \item Al Conacyt por la beca otorgada durante estos cuatro años de doctorado.
    \item A mis pap\'{a}s y mis hermanas por todo su apoyo a\'{u}n en la distancia.
    \item Al Dr.\ Jos\'{e} Mustre no s\'{o}lo por ser un excelente asesor de tesis sino por el inter\'{e}s personal que toma en el bienestar de sus alumnos.
    \item A todos los profesores del Cinvestav M\'{e}rida con los que tuve oportunidad de convivir por sus enseñanzas.
    \item Al personal administrativo del Cinvestav que, en mi experiencia, siempre apoya con gusto y una sonrisa.
    \item A la comunidad de software libre por la enorme cantidad de herramientas que hicieron posible este trabajo.
\end{itemize}

% Resumen

\cleardoublepage
\phantomsection
\addcontentsline{toc}{chapter}{Resumen}
\begin{center}
\textbf{\large Resumen}
\end{center}
En este trabajo revisamos la evidencia de la existencia de distorciones din\'{a}micas en la red cristalina de superconductores basados en cobre. 
La presencia de estas distorciones est\'{a} relacionada con excitaciones polar\'{o}nicas. 
Usando un modelo simple con un hamiltoniano de Peierls-Hubbard mostramos que, para valores intermedios del acoplamiento electr\'{o}n-red, la representaci\'{o}n en espacio real del primer estado excitado reproduce las distorciones din\'{a}micas observadas debajo de la temperatura de aparici\'{o}n del pseudogap $T^*$. 
Las energ\'{i}as de los estados excitados predichas por este modelo muestran efectos isot\'{o}picos muy diferentes dependiendo de la naturaleza del estado excitado. 
Estas diferencias pueden explicar los resultados conflictivos obtenidos por diferentes t\'{e}cnicas, ya que \'{e}stas exploran diferentes excitaciones en la fase del pseudogap. 
Finalmente, discutimos la plausibilidad de interpretar el estado base del pseudogap como una mezcla inhonomogenea de regiones con portadores bipolar\'{o}nicos y part\'{i}culas tipo fermiones quasi-libres a partir de los cuales la superconductividad emerge a $T_c$.

% Abstract

\cleardoublepage
\phantomsection
\addcontentsline{toc}{chapter}{Abstract}
\begin{center}
\textbf{\large Abstract}
\end{center}
We review the evidence for the existence of local dynamical lattice distortions in cuprates in the pseudogap region of the phase diagram. 
The presence of these distortions is related to polaronic excitations. 
Using a simple Peierls-Hubbard Hamiltonian we show that, for intermediate values of the electron-lattice coupling, the real space representation of the first excited state reproduces the observed local lattice distortions below the pseudogap appearance temperature, $T^*$. 
The excited state energies predicted by the model exhibit very different isotopic effects depending on the nature of the particular excited state. 
These differences can explain conflicting results obtained with different techniques, as these probe different excitations of the pseudogap phase. 
Finally, we discuss the plausibility of interpreting the pseudogap ground state as an inhomogeneous mixture of nanoscale regions of bipolaronic carriers and quasi-free fermion like particles from which the superconducting state arises at $T_c$.

% Objective

\cleardoublepage
\phantomsection
\addcontentsline{toc}{chapter}{Objective}
\begin{center}
\textbf{\large Objective}
\end{center}

In this thesis we aim to increase the understanding of high-temperature cuprate superconductors through the study of the polaronic effects present in such materials.
It has been shown that simple Fermi-liquid models relying on the Born-Oppenheimer approximation are inadequate for an accurate description of these materials.
Given the evidence of polaronic behaviour in the O(4)-Cu(1)-O(4) cluster of the YBa$_2$Cu$_3$O$_{7-\delta}$ high-temperature superconductor, we analyze a simple model, coupling charge and lattice degrees of freedom, and explain some observed properties in terms of its lowest excitations.
This model has been used to explain the observed copper-oxigen bond-lenght differences in the O(4)-Cu(1)-O(4) cluster and its optical signatures.
The objective of this thesis is to reproduce those results and expand upon them to include a thorough description of the electronic excitations incluiding the importance of charge-lattice coupling as revealed by the effects of an oxygen isotope substitution.

% Foreword

\cleardoublepage
\phantomsection
\addcontentsline{toc}{chapter}{Foreword}
\begin{center}
\textbf{\large Foreword}

\textit{Reproducibility as a workflow issue}
\end{center}

Throughout the four years it took to develop the work presented in this thesis I feel my effort was directed towards two main objectives.
The first one was to understand the research problem adequately enough to be able to make a significant contribution. 
This effort is, I hope, clearly demonstrated in this manuscript.
The second and, in my opinion, equally important objective is the acquirement of an efficient research methodology.
It is to this methodology that I want to direct a few lines in this foreword.

The core principle I had in mind while developing my own research methodology was that of \textit{reproducibility}.
It is my intention that all the details about my research can be easily reproduced by any interested reader.
This should be a feature about every scientific work but, sadly, reproducibility is not always facilitated.
At times, it is downright impossible.
To achieve this objecive in my work I leaned towards an open approach, as vibrantly encouraged by the \textit{Open Science} community, based on internet tools.
I tried many existing web-based technologies but none of them suited my exact needs, so I developed my own tool which is called \textit{Research Engine}\footnote{Reachable at https://research-engine.appspot.com}.
All the information, datasets and computer algorithms required to reproduce the work presented here (as well as future updates) can be obtained from the project's homepage in Research Engine: https://research-engine.appspot.com/37001. 
Should any reader of this thesis need any clarification about my work, please feel free to open a thread in the project's forum.
I will answer any inquiry to the best of my ability as soon as possible.

Although the Research Engine tool is now well suited for my daily research endeavours, it still lacks an utility to easily write large documents, such as this thesis.
Keeping true to the reproducibility principle, I am making all the \LaTeX\ files, datasets and plotting scripts needed to generate this document in a single and easily accessible place: https://github.com/andresgsaravia/PhD-thesis under a CC0 license\footnote{See https://creativecommons.org/publicdomain/zero/1.0 for details.
Basically this means that anyone can use the present work for any purpose without the need for a permission from the authors.}.
