% Aknowledgements

\cleardoublepage
\phantomsection
\addcontentsline{toc}{chapter}{Aknowledgments}

\begin{center}
\textbf{\large Agradecimientos}
\end{center}
\vspace{3cm}

\begin{itemize}
    \item Al Conacyt por la beca otorgada durante estos cuatro años de doctorado.
    \item A mis pap\'{a}s y mis hermanas por todo su apoyo a\'{u}n en la distancia.
    \item Al Dr. Jos\'{e} Mustre no s\'{o}lo por ser un excelente asesor de tesis sino por el inter\'{e}s personal que toma en el bienestar de sus alumnos.
    \item A todos los profesores del Cinvestav M\'{e}rida con los que tuve oportunidad de convivir por sus enseñanzas.
    \item Al personal administrativo del Cinvestav que, en mi experiencia, siempre apoya con gusto y una sonrisa.
    \item A la comunidad de software libre por la enorme cantidad de herramientas que hicieron posible este trabajo.
\end{itemize}

% Resumen

\cleardoublepage
\phantomsection
\addcontentsline{toc}{chapter}{Resumen}
\begin{center}
\textbf{\large Resumen}
\end{center}
En este trabajo revisamos la evidencia de la existencia de distorciones din\'{a}micas en la red cristalina de superconductores basados en cobre. 
La presencia de estas distorciones est\'{a} relacionada con excitaciones polar\'{o}nicas. 
Usando un modelo simple con un hamiltoniano de Peierls-Hubbard mostramos que, para valores intermedios del acoplamiento electr\'{o}n-red, la representaci\'{o}n en espacio real del primer estado excitado reproduce las distorciones din\'{a}micas observadas debajo de la temperatura de aparici\'{o}n del pseudogap $T^*$. 
Las energ\'{i}as de los estados excitados predecidas por este modelo muestran efectos isot\'{o}picos muy diferentes dependiendo de la naturaleza del estado excitado. 
Estas diferencias pueden explicar los resultados conflictivos obtenidos por diferentes t\'{e}cnicas, ya que \'{e}stas exploran diferentes excitaciones en la fase del pseudogap. 
La plausibilidad de interpretar el estado base del pseudogap como una mezcla inhonomog\'{e}nea de regiones con portadores bipolar\'{o}nicos y part\'{i}culas tipo fermiones quasi-libres a partir de los cuales la superconductividad emerge a $T_c$ es discutida.

% Abstract

\cleardoublepage
\phantomsection
\addcontentsline{toc}{chapter}{Abstract}
\begin{center}
\textbf{\large Abstract}
\end{center}
Evidence for the existence of local dynamical lattice distortions in cuprates in the pseudogap region of the phase diagram is reviewed. 
The presence of these distortions is related to polaronic excitations. 
Using a simple Peierls-Hubbard Hamiltonian we show that, for intermediate values of the electron-lattice coupling, the real space representation of the first excited state reproduces the observed local lattice distortions below the pseudogap appearance temperature, $T^*$. 
The excited state energies predicted by the model exhibit very different isotopic effects depending on the nature of the particular excited state. 
These differences can explain conflicting results obtained with different techniques, as these probe different excitations of the pseudogap phase. 
The plausibility of interpreting the pseudogap ground state as an inhomogeneous mixture of nanoscale regions of bipolaronic carriers and quasi-free fermion like particles from which the superconducting state arises at $T_c$ is discussed.

% Objective

\cleardoublepage
\phantomsection
\addcontentsline{toc}{chapter}{Objective}
\begin{center}
\textbf{\large Objective}
\end{center}
