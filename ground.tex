\chapter{Cluster distortion and bipolaron formation}
\label{chap:ground}

As discussed in section \ref{sec:lattice-distortions}, from the ground state of the model hamiltonian (\ref{eq:full-hamiltonian}) we can calculate, using (\ref{eq:dvsuir}), the split CuO distance for different values of the coupling parameter $\lambda_{ir}$.
In section \ref{sec:grd-phonon-proj} of this chapter we reproduce the previous work by Mustre de Le\'{o}n \textit{et al.} \cite{MustredeLeon1992} which, from this calculation, determied that the relevant couplings are in the intermediate regime.

Since the model hamiltonian (\ref{eq:full-hamiltonian}) exhibits polaronic behaviour only for electron-lattice couplings greater than zero, we can identify the difference between the ground state energy in the absence of electron-lattice interaction ($\lambda_{ir}=0$) and its value for the coupling value where this polaronic behaviour sets in as the bipolaron binding enerygy.
In section \ref{sec:grd-binding-energy} we calculate this binding energy comparing it with experimental results.
We also calculate the isotopic shift, as defined in (\ref{eq:isot-shift-def-grd}), for the oxygen substitution$^{16}$O$\rightarrow ^{18}$O as a function of $\lambda_{ir}$.

\section{Cu(1)-O(4) local distortion}
\label{sec:grd-phonon-proj}

The wavefunction projection into phonon coordinates (\ref{eq:phonon-coord-projection}) gives information about the coodinates of the three atoms in the CuO$_2$ cluster.
In particular, the phonon coordinate $u_{ir}$\footnote{The phonon coordinates are  defined in (\ref{eq:uR}, \ref{eq:uir}).} is directly proportional to the CuO distance distortion, $d$, as shown in (\ref{eq:dvsuir}).
Figure \ref{fig:ph-ground} shows the ground state projected into phonon coordinates $(u_R,u_{ir})$ for three representative electron-lattice coupling values $\lambda_{ir}=0, 0.13$ and 0.25 eV.
From the $\lambda_{ir}=0$ case the wavefunction's projection has a maximum at $u_{ir}=0$ which implies, as expected, an absence of lattice distortions.
With increasing coupling, two maxima start to develop giving rise to a lattice distortion.
We can notice that in the middle coupling regime, exemplified in Fig. \ref{fig:ph-ground} by $\lambda_{ir}=0.13$ eV, that the two peaks are not fully separated.
Thus the system can \textit{tunnel} between both possible configurations with the longer distance bonding the first oxygen or the second.
This observation means that the distortion is dynamical. 
For greater coupling values the two peaks are completely separated and the distortion become static.

\begin{figure}[ht!]
  \centering
  \includegraphics[width=1.0\textwidth]{images/ph-ground.png}
  \caption{Ground state's projection into phonon coordinates for three different values of the coupling parameter $\lambda_{ir}$.}
  \label{fig:ph-ground}
\end{figure}

Since the projection into $u_R$ does not change, we also considered the projections with $u_R=0$ and variable $u_{ir}$ for different coupling values.
The left panel of figure \ref{fig:uir-vs-coupl} shows a plot of this comparison.
From here we can see that the distortion only sets in for $\lambda_{ir} > \sim 0.12$ eV and becomes static (the two peaks are fully separated) for $\lambda_{ir} > \sim 0.16$ eV.
In the right panel of figure \ref{fig:uir-vs-coupl} we show the cluster distortion $d$ calculated from the maxima of the previous projection.
Here we can see that the observed cluster distortion of $\sim 0.13$ \AA\ \cite{MustredeLeon1990} is reproduced in the intermedite coupling values. 
In particular, $\lambda_{ir}=0.1263$ eV reproduces that distortion.

\begin{figure}[ht!]
  \centering
  \includegraphics[width=1.0\textwidth]{images/uir-vs-coupl-d.png}
  \caption{Projection into phonon coordinates with $u_R=0$ (left panel) and calculated cluster distortion $d$ (right panel) for different $\lambda_{ir}$ coupling values.}
  \label{fig:uir-vs-coupl}
\end{figure}

\section{Bipolaron binding energy}
\label{sec:grd-binding-energy}

The bipolaron formation energy is $E_p = E(\lambda_{ir}=0)- E(\lambda_{ir}=0.13$ eV$) \sim 42$ meV and corresponds to the bipolaron binding energy. This value compares favorably with the value obtained from femtosecond time-domain spectroscopy ($E_p \sim 45$ meV for YBa$_2$Cu$_3$O$_7$ \cite{Demsar1999}. We also find that if we consider a smaller electron-lattice coupling such that the distortion is 0.08 $\AA$, as that observed for in plane Cu(2)-O in La$_{1.85}$Sr$_{0.15}$CuO$_4$ \cite{Bianconi1996}, we obtain $E_p \sim 32$ meV, which is also comparable to estimates for the pseudogap formation energy in this system \cite{Kusar2005}.

\section{Isotopic effects in bipolaron formation}
\label{sec:grd-isotopic}

For the ground state we define the energy isotopic shift $\Delta_g$ in a similar way but with the energies measured relative to the uncoupled system (that is, the system with $\lambda_{ir}=\lambda_R=0$) (see \ref{eq:isot-shift-def-grd}):

\begin{equation}
  \Delta_g = \frac{\Delta E_g(^{16}O)- \Delta E_g(^{18}O)}{\Delta E_g(^{16}O)} \times 100 
\end{equation}

where $\Delta E_g \equiv E_g - E_g(\lambda_{ir}=0, \lambda_R=0)$.\footnote{To calculate this energies we need to take into account the \textit{zero-point energy} in the phononic part $H_{ph}$ from (\ref{eq:full-hamiltonian}) which is not explicitly included in previous publications.}

The following figure plots this isotopic shift:

\begin{figure}[ht!]
  \centering
  \includegraphics[width=0.8\textwidth]{images/isot_polaron_formation.jpg}
  \caption{Isotopic shift of the polaron formation energy.}
  \label{fig:isot_polaron_formation}
\end{figure}

Although $\Delta_g$ doesn't change sign, as the isotopic shift of the polaron tunneling state, it shows a maximum in the middle coupling regime reminiscent of the maximums, minimums or inflection points in the isotopic shifts for all the other excitations.